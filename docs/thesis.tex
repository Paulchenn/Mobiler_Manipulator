%%%%%%%%%%%%%%%%%%%%%%%%%%%%%%%%%%%%%%%%%%%%%%%%%%
%
% Template for group Projects by students of Hochschule Karlsruhe (University of Applied Sciences)
% 
% created by: Prof. Dr.-Ing. Thomas Hollstein (for Frakfurt University of Applied Sciences)
% adapted by: Paul Glaser, B. Eng (for Hochschule Karlsruhe)
% 
% Last revision of Prof. Dr.-Ing. T. Hollstein: 22.06.2023
% Last revision of P. Glaser: 02.04.2025
%
%%%%%%%%%%%%%%%%%%%%%%%%%%%%%%%%%%%%%%%%%%%%%%%%%%




%%%%% Basic formatting %%%%%
\documentclass[
	%twoside, 
	%openright,
	titlepage,
	numbers=noenddot,
	headinclude,
	%1headlines,
	footinclude=true,
	%cleardoublepage=empty,
	%BCOR=5mm,
	fontsize=12pt,%11pt,
	paper=a4paper,
	%letterpaper
	ngerman,
	american,
	%table
 ]
%% Document type: (specifies basic formatting guidelines)
%{report}
{scrreprt}
%{book}

% Colored Text
\usepackage{xcolor}
\usepackage{colortbl}

\parindent = 0pt % Do not indent new paragraphs
\parskip = 1ex % New paragraphs: 1/2 line spacing




%%%%%%%%%%%%%%%
%%%%% Settings %%%%%
%%%%%%%%%%%%%%%




%%%%% integrate packages to be used %%%%%


%% For ToDos
%\usepackage{todonotes}


%% Package Comments
\usepackage{comment}
\usepackage[textwidth=3cm]{todonotes} %[colorinlistoftodos]
\newcommand{\mycomment}[1]{\todo[inline,linecolor=red,backgroundcolor=yellow!25,bordercolor=green,caption={}]{?? TODO: #1}}


%% Package Language support:
\usepackage[utf8]{inputenc} % Set input encoding to UTF-8
\usepackage[T1]{fontenc}    % Set font encoding to T1
\usepackage[ngerman]{babel}
%\usepackage[german]{babel}
\usepackage[autostyle]{csquotes}
%\usepackage{csquotes}


%% Package Multirow allows a box in a table to span multiple rows
\usepackage{multirow}
\usepackage{array}
\usepackage{ragged2e}
\usepackage{booktabs}
\usepackage{arydshln}
% Examples: https://texblog.org/2012/12/21/multi-column-and-multi-row-cells-in-latex-tables/

%% Joining fields in tables:
%\multicolumn{number cols}{align}{text} % align: l,c,r
%\multirow{number rows}{width}{text}


%% Sans serif text style for the entire document
\RequirePackage[sfdefault,lf]{carlito}
\usepackage{lmodern}
% \RequirePackage[T1]{fontenc}
% to imitate Calibri:
\renewcommand*\familydefault{\sfdefault} % Base font of the document is to be sans serif
\usepackage{lipsum}
\usepackage{listings}


%% Embed graphics
\usepackage{float}
\usepackage{graphicx}
\usepackage{subcaption}
\usepackage{lscape}
\usepackage{svg}
\usepackage{pgf-pie}
\usepackage{transparent}
\usepackage{pdfpages}
\usepackage{pgfplots}
\pgfplotsset{compat=newest}
\pgfplotsset{plot coordinates/math parser=false}
\usetikzlibrary{patterns,arrows,shapes,positioning,shadows,trees,calc,positioning,external}
\usepackage{tikz}
%\tikzexternalize[prefix=Figures/tikzexternalize_cache/,optimize command away=\includepdf]
\tikzexternaldisable
% https://golatex.de//wiki/%5cincludegraphics

%% For tree diagrams
\tikzset{
	basic/.style  = {draw, rectangle},
	root/.style   = {basic, rounded corners=2pt, thin, align=center},
	level 2/.style = {basic, rounded corners=6pt, thin,align=center, text width=5em},
	level 3/.style = {basic, align=left, draw=none, text width=60}
}


%% Calculations
%https://tex.stackexchange.com/questions/30081/how-can-i-sum-two-values-and-store-the-result-in-other-variable
\usepackage{tikz}
\usetikzlibrary{math}
\usepackage{amssymb}
\usepackage[fleqn,reqno]{amsmath}
\usepackage{mathtools}


%% fractures
\usepackage{nicefrac} % For comparison
\usepackage{xfrac}    % Works better with other fonts
\usepackage{romannum}


%% More flexible tables
\usepackage{tabularx}
% \usepackage{xltabular}
\def\tabularxcolumn#1{m{#1}}
\usepackage{enumitem}
%\usepackage{enumerate}
\usepackage{layouts}
\usepackage{listliketab}
\usepackage{rotating} 


%% Boxes for Lickert scales
\usepackage{wasysym}
\newcommand\insq[1]{%
	\Square\ #1\quad%
}


% Strikethrough text
\usepackage[normalem]{ulem} % mit \sout


%% Use total page count
\usepackage{totpages}


%%%%% ----->>>>> CONFLICTS WITH ENUMITEM <<<<<----- %%%%%
%%% Advanced formats for lists/enumerations
%\usepackage{paralist}
%\setdefaultitem{}{\textbullet}{$\star$}{} % Default items for the four possible nesting levels
%%%%% ----->>>>> CONFLICTS WITH ENUMITEM <<<<<----- %%%%%


%% Hyperlinks
% https://www.overleaf.com/learn/latex/Hyperlinks
\usepackage{hyperref}
\hypersetup{
	%hyperindex=true,
	%linktocpage=true, % Page number linked instead of title
	colorlinks=true,
	linkcolor=black,
	filecolor=black,      
	urlcolor=black,
	pdftitle={Thesis},
	citecolor=black,
	pdfpagemode=UseNone, % No automatic view setting
	pdfstartpage=1,      % Starts on the first page
	pdfstartview=Fit     % Shows the entire page in the window
}


%% Include acronym definitions
\usepackage[printonlyused]{acronym}
\usepackage{tocloft}
%\input{TeXFiles/005_Acronyms}


%% Include glossary
\usepackage{glossaries}
\makeglossaries

%% Glossareintraege
\newglossaryentry{Pruefung}{
	name=Prüfung,
	description={Eine Prüfung enthält mehrere Einzelmessungen und kann entweder Batch-Sequenziell oder DUT-Sequenziell durchgeführt werden}
}
%\usepackage[record]{bib2gls}
%\addbibresource{Glossar.bib}



%%%%% Adjust page geometry %%%%%
% https://tex.stackexchange.com/questions/344241/logo-as-header-using-fancyhdr-package
\usepackage{geometry}
\geometry{verbose,
	bmargin=2cm,
	lmargin=2.8cm,
	rmargin=3.5cm
	%footskip=-25pt
}


%% Calculate the amount of the top margin
\newlength\mytopmargin
\newsavebox{\headbox}\savebox{\headbox}{
	%\includegraphics[width=0.3\textwidth]{Figures/xxx.png} \hfill
	\raisebox{-1ex}{\includegraphics[width=0.1\textwidth]{Figures/png/HKA_Logo.png}}
}    
\setlength{\mytopmargin}{\totalheightof{\usebox{\headbox}}+2cm}
\geometry{verbose,
	tmargin=\mytopmargin,
	headheight=1.1\mytopmargin,
	footskip=6ex
}

%% Line spacing for the entire document (1.0 = normal value)
\renewcommand{\baselinestretch}{1.0}

%% Table of contents up to the 2nd numbering level
\setcounter{tocdepth}{1}



%%%%% Designing headers and footers %%%%%
% https://tex.stackexchange.com/questions/344241/logo-as-header-using-fancyhdr-package
\usepackage{fancyhdr}
\pagestyle{fancy}  % Eigener Seitenstil
\fancyhf{}         % Alle Kopf- und Fußzeilenfelder bereinigen
%\fancyhead[L]{} % Kopfzeile links
\fancyhead[l]{\leftmark
	%\makebox[0.3\textwidth]{\includegraphics[width=0.3\textwidth]{Figures/xxx.png}}
}  
% \fancyhead[c]{\hspace*{0.15\textwidth}\rightmark}
%\fancyhead[C]{\usebox\headbox}                        % Centered header
\fancyhead[R]{
%	\makebox[0.2\textwidth]{\raisebox{-1ex}{\hspace*{14ex}\includegraphics[width=0.1\textwidth]{Figures/png/HKA_Logo.png}}}
	\makebox[0.2\textwidth]{\raisebox{-1ex}{\hspace*{8ex}\includegraphics[width=0.1\textwidth]{Figures/png/HKA_Logo.png}}}
}  % Kopfzeile rechts
\renewcommand{\headrulewidth}{0.4pt} % Upper dividing line
%\fancyfoot[L]{\today}
\fancyfoot[C]{\ThesisTitleShort} 
%\fancyfoot[R]{Seite \thepage ~von \ref{TotPages}}  % Page number
\fancyfoot[R]{\thepage} % Page number 
\renewcommand{\footrulewidth}{0.4pt} % Lower dividing line
\setlength{\mytopmargin}{\totalheightof{\usebox\headbox} +2cm}

%% Difference between even/odd pages
%\fancyhead[OR]{} % "O" stands for "odd", i.e. odd sides
%\fancyhead[ER]{} % "E" for "even", i.e. straight sides



%%%%% HKA CI colours %%%%%
%% CI-Colour HKA MMT-Blue
\definecolor{HKA_MMTblue}{RGB}{58, 104, 170}


%% Color section titles according to CI colors
\usepackage{titlesec}
\titleformat{\section}
{\color{HKA_MMTblue}\normalfont\Large\bfseries} %Titel
{\color{HKA_MMTblue}\thesection}{1em}{}
\titleformat{\subsection}
{\color{HKA_MMTblue}\normalfont\large\bfseries} %Titel
{\color{HKA_MMTblue}\thesubsection}{1em}{}
\titleformat{\subsubsection}
{\color{black}\normalfont\bfseries} %Titel
{\color{black}\thesubsection}{0.5em}{}


%%%%% modern BibLaTeX with biber for bibliography %%%%%
% https://golatex.de/viewtopic.php?t=13917
\usepackage[style=ieee, backend=biber, sorting=none, natbib=true]{biblatex}
%\usepackage[style=ieee-alphabetic,backend=biber, natbib=true]{biblatex}
%\usepackage[sorting=none, style=numeric,backend=biber, natbib=true]{biblatex}
%\usepackage[style=apa, backend=biber, natbib=true, sorting=nyt, sortcites=false]{biblatex}
%\usepackage[style=numeric, backend=biber, natbib=true, sorting=nyt, sortcites=false]{biblatex}
%\usepackage[ style=numeric, backend=biber, natbib=true]{biblatex}

\addbibresource{bibliography.bib}

% Always show short author if available
\makeatletter
\def\cbx@apa@ifnamesaved{\@firstoftwo}
\makeatother




%%%%%%%%%%%%%%%%%%%%%%%%%%%
%%%%% the actual document begins here %%%%%
%%%%%%%%%%%%%%%%%%%%%%%%%%%




\begin{document}
	%% Integrate settings
	\newcommand{\NameA}{Paul Glaser, B. Eng}
\newcommand{\StudentIdA}{glpa1013 / Matrikelnr. 100215}
\newcommand{\NameB}{Tim Schäfer, B. Eng}
\newcommand{\StudentIdB}{scti1057 / Matrikelnr. 71606}
\newcommand{\NameC}{Felix Wietschel, B. Eng}
\newcommand{\StudentIdC}{scti1057 / Matrikelnr. 74940}
\newcommand{\ThesisTitle}{Projektaufgabe Roboterprogrammierung}
\newcommand{\ThesisTitleShort}{Mobiler Manipulator}
\newcommand{\ThesisSubtitle}{Mobiler Manipulator}
\newcommand{\Subject}{Roboterprogrammierung --~Wintersemester 25/26}
\newcommand{\Supervisor}{Prof.~Dr.-Ing. Björn Hein}
\newcommand{\Faculty}{Fakultät für Maschinenbau und Mechatronik}
\newcommand{\University}{Hochschule Karlsruhe}
\newcommand{\UniversityLocation}{Karlsruhe}
\newcommand{\ThesisDeliveryDate}{20.~Januar.~2026}
\newcommand{\CompanyName}{NoCompanyInvolved}  % don't modify this line, if no company is involved

%\renewcommand{\theenumi}{\textbf{(\alph*{enumi})}} % Setze die Aufzählung auf fett


\newcommand{\VarPicWidthA}{\textwidth}
\newcommand{\VarPicWidthB}{0.9\textwidth}
\newcommand{\VarPicWidthC}{0.8\textwidth}
\newcommand{\VarPicWidthD}{0.7\textwidth}
\newcommand{\VarPicWidthE}{0.6\textwidth}
\newcommand{\VarPicWidthF}{0.49\textwidth}
\newcommand{\VarPicWidthG}{0.32\textwidth}
\newcommand{\VarPicWidthH}{0.24\textwidth}
\newcommand{\VarPicWidthI}{0.2\textwidth}
\newcommand{\VarPicWidthJ}{0.469\textwidth}
\newcommand{\VarPicWidthK}{0.45\textwidth}

\newcolumntype{X}{>{\raggedleft\arraybackslash}X} % zentriert
\newcolumntype{Y}{>{\centering\arraybackslash}X} % zentriert
\newcolumntype{Z}{>{\raggedright\arraybackslash}X} % linksbündig

%\definecolor{blue40} {rgb}{0.0000, 0.3843, 0.6039}
\definecolor{blue50} {rgb}{0.0000, 0.4824, 0.7529}
%\definecolor{blue55} {rgb}{0.0000, 0.5333, 0.8314}
\definecolor{turq50} {rgb}{0.0941, 0.5137, 0.4941}
%\definecolor{green40}{rgb}{0.0000, 0.4235, 0.2275}
\definecolor{green50}{rgb}{0.0000, 0.5333, 0.2902}
%\definecolor{green55}{rgb}{0.1294, 0.5843, 0.3412}
\definecolor{purp40} {rgb}{0.6196, 0.1569, 0.5882}
%\definecolor{red40}  {rgb}{0.7451, 0.0000, 0.0157}
\definecolor{red50}  {rgb}{0.9294, 0.0000, 0.0725}
%\definecolor{red55}  {rgb}{1.0000, 0.1294, 0.1412}
\definecolor{gray40} {rgb}{0.3490, 0.3686, 0.3843}%
\definecolor{gray70} {rgb}{0.6431, 0.6706, 0.7020}%

\newcommand{\ti}[1]{_\mathrm{#1}}
\newcommand\tab[1][1cm]{\hspace*{#1}}

\newcommand{\abs}[1]{\ensuremath{\left\vert#1\right\vert}}
\newcommand{\rdown}[1]{\ensuremath{\left\lfloor#1\right\rfloor}}
%\DeclarePairedDelimiter\abs{\lvert}{\rvert}

% Abstand für den Rahmen auf 0pt setzen
\setlength{\fboxsep}{0pt}

\newlist{listNorm}{enumerate}{4}
\setlist[listNorm,1]{label={\scriptsize\(\bullet\)}}
\setlist[listNorm,2]{label={\scriptsize-}}
\newlist{listArab}{enumerate}{4}
\setlist[listArab]{label*=\arabic*.}
\newlist{listAlph}{enumerate}{4}
\setlist[listAlph]{label*=(\alph*)}
\newlist{listAlphB}{enumerate}{4}
\setlist[listAlphB]{label*=\textbf{(\alph*)}}

% Definieren von Stilen für die Formen und Pfeile im Diagramm
\tikzstyle{startstop} 		= [rectangle, rounded corners, minimum width=3cm, minimum height=1cm,text centered, draw=black]
\tikzstyle{process}			= [rectangle, rounded corners, text centered, draw=black]
\tikzstyle{info} 			= [rectangle, text centered, draw=none]
\tikzstyle{decision} 		= [diamond, minimum width=3cm, minimum height=1cm, text centered, draw=black]
\tikzstyle{arrow} 			= [thick,->,>=stealth]
\tikzstyle{arrow2} 			= [thick,<->,>=stealth]
\tikzstyle{vorangeganen} 	= [rectangle, rounded corners, text centered, draw=black, fill=blue55, fill opacity=0.1]
\tikzstyle{hier} 			= [rectangle, rounded corners, text centered, draw=black]
\tikzstyle{weiterführen} 	= [rectangle, rounded corners, text centered, draw=black, fill=green55, fill opacity=0.1]

% For code
\definecolor{codegray}{rgb}{0.5,0.5,0.5}
\definecolor{codegreen}{rgb}{0,0.6,0}
\definecolor{backcolour}{rgb}{0.95,0.95,0.92}

\lstdefinestyle{pythonstyle}{
    backgroundcolor=\color{backcolour},   
    commentstyle=\color{codegreen},
    keywordstyle=\color{magenta},
    numberstyle=\tiny\color{codegray},
    stringstyle=\color{HKA_MMTblue}, % Nutzung der HKA Farbe für Strings
    basicstyle=\ttfamily\footnotesize,
    breakatwhitespace=false,         
    breaklines=true,                 
    captionpos=b,                    
    keepspaces=true,                 
    numbers=left,                    
    numbersep=5pt,                  
    showspaces=false,                
    showstringspaces=false,
    showtabs=false,                  
    tabsize=2,
    language=Python
}
\lstset{style=pythonstyle}


%%%%%%%%%%%%%%%%%%%%% FOR ENGLISH LANGUAGE THESIS: %%%%%%%%%%%%%%%%%%%%

% By default the thesis language is german
% if you want to set it to ENGLISH, then UNCOMMENT the FOLLOWING LINE by removing the leading "%":

%\newcommand*{\ThesisLanguageIsEnglish}{}

%% End of File
	\sloppy % Avoid formatting overhangs at the end of lines
	% https://latexref.xyz/_005cfussy-_0026-_005csloppy.html

	\frenchspacing  % A space after the end of a sentence
	%https://texwelt.de/fragen/1154/was-ist-french-spacing-was-macht-frenchspacing

	%Default is \flushbottom, which means all sides are stretched so that they are the same height
	%\raggedbottom   


	%% Set German according to the new spelling rules as the default language for the document
	\ifdefined\ThesisLanguageIsEnglish
		\selectlanguage{american}
	\else
		\selectlanguage{ngerman} % ngerman, american
	\fi
	%

	%\renewcommand*{\bibname}{new name}
	%\setbibpreamble{}

	%% Set numbering depth
	% 2: to subsection (standard)
	% 3: to subsubsection
	\setcounter{secnumdepth}{2} % sets the numbering depth


	% Display page without headers and footers
	%\renewcommand{\thepage}{\Roman{page}}
	\pagestyle{plain} 

	%\listoftodos[ToDo's]
	

	% The title page and, if applicable, the embargo notice are included here
	%*******************************************************
% Titlepage
%*******************************************************
%%%
%%% title page (german)
%%%
\thispagestyle{empty}
\pdfbookmark[0]{Titelblatt}{title}

\begin{titlepage}
	
	% Fra-UAS Logo
	\vspace*{-2,5cm}
  	\begin{center}
    		\includegraphics[width=7.7cm]{Figures/png/HKA_Logo.png} \\ 
  	\end{center}

	% Name FH und Fakultaet
  	\begin{center}
		\vspace{0.1cm}
		\LARGE \textbf{\University}\\
		\vspace{0.4cm}
		\Large -- \Faculty --
	\end{center}
	
	\vfill
	
	% Titel (--> Bitte in TeXFiles/000_Settings eintragen)
	\begin{center}
		\huge \textbf{\ThesisTitle}\\
		\vspace{0.4cm}
		\LARGE \ThesisSubtitle
	\end{center}
	
	\vfill
	
	% Untertitel (--> Bitte in TeXFiles/000_Settings eintragen)
	\ifdefined\ThesisLanguageIsEnglish
		\begin{center}
			\Large Project documentation for\\
			\vspace{0.3cm}
			\Large \Subject
		\end{center}
	\else
		\begin{center}
			\Large Projektdokumentation zum\\
			\vspace{0.3cm}
			\Large \Subject   %%%%% >>>>> Bitte in TeXFiles/000_Settings eintragen
		\end{center}
	\fi
	
	\vfill
	
	% Datum, Name, Matrikelnummer (--> Bitte in TeXFiles/000_Settings eintragen)
	\ifdefined\ThesisLanguageIsEnglish 
  		\begin{center}
    			\Large submitted on \ThesisDeliveryDate\ on\\
    			\vspace{0.3cm}
    			\Large \textbf{\NameA}\\
    			\vspace{0.3cm}
    			\normalsize Student ID: \StudentIdA
  		\end{center}
  	\else
		\begin{center}
			\Large vorgelegt am 30. September 2025 von\\
			\vspace{0.3cm}
			\Large \textbf{\NameA} (\StudentIdA)\\
			\vspace{0.3cm}
			\Large \textbf{\NameB} (\StudentIdB)\\
			\vspace{0.3cm}
			\Large \textbf{\NameC} (\StudentIdC)\\
		\end{center}
  	\fi
	
	\vfill
	
	% Betreuer (--> Bitte in TeXFiles/000_Settings eintragen)
	\ifdefined\ThesisLanguageIsEnglish 
		\begin{center}
			\begin{tabular}{lll}
				First Supervisor    & : & \Supervisor \\     %%%%% >>>>> Bitte in TeXFiles/000_Settings eintragen
				Second Supervisor & : & \CoSupervisor\\    %%%%% >>>>> Bitte in TeXFiles/000_Settings eintragen
			\end{tabular}
		\end{center}
	\else
		\begin{center}
			\begin{tabular}{lll}
				Erstprüfer    		& : & \Supervisor \\     %%%%% >>>>> Bitte in TeXFiles/000_Settings eintragen
				% Zweitprüfer			& : & \CoSupervisor\\    %%%%% >>>>> Bitte in TeXFiles/000_Settings eintragen
				% Projektkoordinator	& : & \CompSupervisor\\    %%%%% >>>>> Bitte in TeXFiles/000_Settings eintragen
				\ifdefstring{\CompanyName}{NoCompanyInvolved}{}{
					Betreuer Fa. \CompanyName 	& : & \CompSupervisor\\    %%%%% >>>>> Bitte in TeXFiles/000_Settings eintragen
				}
			\end{tabular}
		\end{center} 
	\fi
	
	\newpage
	
\end{titlepage}
 
 %% End of File
	\input{TeXFiles/002_NDNotice}
	\newpage


	%% Since we created the title manually, the following command is omitted
	% \maketitle

	% \clearpage

	%\pagestyle{headings} 
	\pagestyle{fancy}   % Display page with headers and footers
	\pagenumbering{roman}  % Switch to page numbers in Roman numerals


	%% Create directories
	% Table of content
	\tableofcontents
	\newpage
	% Table of Figures
	\listoffigures
	\newpage
	% Table of tables
	\listoftables
	% Glossary
	\ifdefined\ThesisLanguageIsEnglish
		\printglossary[title=Glossary, toctitle=Glossary]
	\else
	%		\printglossary[title=Glossar, toctitle=Glossar]
		\printglossaries
	\fi
	% Acronyms
	% Allgemeine Akronyme
\chapter*{Abkürzungsverzeichnis}
\begin{acronym}[LONG]
	\acro{DUT}{Device under Test}
\end{acronym}
%% End of File %%
	
	\clearpage
	
	%% Switch to page numbers in Arabic format
	\pagenumbering{arabic}
	
	
	%% Integrating the files for the individual sections
	% If you use \include instead of \input, new sections always start on a new page

	\fancyfoot[R]{\thepage}  % Page number on
	
	% \input{TeXFiles/010_Kurzfassung}
	\chapter{Einleitung}
\label{ch:einleitung}
In diesem Projekt wird ein mobiler Manipulator simuliert, der aus einer beweglichen Basis (3 Freiheitsgrade) und einem Arm mit zwei rotatorischen Gelenken besteht (insgesamt 5 Freiheitsgrade). Ziel der Arbeit ist die Implementierung eines robusten Kollisionsprüfers (Collision Checker) sowie die Planung und Evaluierung von kollisionsfreien Bewegungsbahnen unter Verwendung von Sampling-basierten Algorithmen. Im Gegensatz zu exakten Verfahren, die den gesamten Konfigurationsraum explizit berechnen müssten, generieren diese Methoden zufällige Stichproben (Samples) der Roboterpositionen, um probabilistisch eine kollisionsfreie Roadmap (einen Graphen) zu erstellen. In dieser Arbeit werden speziell die Varianten 
\begin{listNorm}
    \item \textbf{\ac{LazyPRM}} und
    \item \textbf{\ac{VisibilityPRM}}
\end{listNorm}
betrachtet.
Der erste Teil dieses Berichts, \textbf{Kapitel~\ref{ch:evaluation},~\nameref{ch:evaluation}}, beschreibt die Implementierung des Kollisionsprüfers und die Durchführung von Benchmark-Tests zur Evaluierung der Planungsverfahren in verschiedenen Szenarien.
Zusätzlich wird im zweiten Teil, \textbf{Kapitel~\ref{ch:pickplace},~\nameref{ch:pickplace}}, die Erweiterung zu einer Pick-And-Place Anwendung vorgestellt sowie die Ergebnisse diskutiert. 
Im dritten Teil, \textbf{Kapitel~\ref{ch:theorie},~\nameref{ch:theorie}}, folgen theoretische Konzepte zur Erweiterung des Systems um translatorische Gelenke sowie Möglichkeiten zur Bahnoptimierung.

Die Aufgabe gliedert sich in folgende Teilbereiche:
\begin{listArab}
    \item Implementierung eines \textit{CollisionCheckers} für einen planaren Roboter (Basis + 2 Links).
    \item Benchmarking verschiedener Planungsverfahren (\ac{LazyPRM} vs. \ac{VisibilityPRM}) in unterschiedlichen Szenarien.
    \item Durchführung einer Pick-and-Place Aufgabe, bei der ein Hindernis gegriffen und transportiert wird.
    \item Theoretische Betrachtung von Systemerweiterungen (translatorische Gelenke und Pfadglättung).
\end{listArab}

Der vollständige Quellcode dieses Projekts ist im entsprechenden Git-Repository hinterlegt: \url{https://github.com/Paulchenn/Mobiler_Manipulator}. Für detaillierte Anweisungen zur Installation und Verwendung wird auf die dort befindliche \texttt{README}-Datei verwiesen.

	\chapter{Implementierung und Evaluation}
\label{ch:evaluation}

\section{Der CollisionChecker}
\label{sec:collision}

Die Basis für alle Planungsverfahren ist eine zuverlässige Kollisionserkennung. Der Roboter wird hierbei modelliert durch eine rechteckige Basis und einen Arm mit zwei Segmenten.

\subsection{Modellierung}
Die Geometrie des Roboters wird über eine Liste definiert, welche die Längen und Dicken der Segmente sowie die Gelenkwinkelgrenzen enthält. Ein typisches Segment ist beispielsweise definiert als:
\begin{lstlisting}
# Beispiel eines Listenelements: [Laenge, Dicke, [MinWinkel, MaxWinkel]]
segment = [5.1, 1, [-3.14, 3.14]] 
\end{lstlisting}

\subsection{Umgang mit Eigenkollisionen}
Eine besondere Herausforderung stellt die Eigenkollision (Self-Collision) dar. Hierbei muss verhindert werden, dass der Arm des Roboters die eigene Basis oder andere Armsegmente durchdringt. Dies wurde implementiert, indem...
% Hier kurz erklären, wie du das gelöst hast (z.B. Prüfung aller Segment-Paare).

\section{Benchmarking der Planer}
Zur Bewertung der Algorithmen \textit{LazyPRM} und \textit{VisibilityPRM} wurden 5 Benchmark-Umgebungen mit steigendem Schwierigkeitsgrad erstellt.

\subsection{Ergebnisse}
Wie im Notebook \texttt{IP-X-1-Automated\_PlanerTest.ipynb} analysiert, zeigen sich folgende Unterschiede:
\begin{itemize}
    \item \textbf{LazyPRM}: Zeigt Stärken in...
    \item \textbf{VisibilityPRM}: Ist effizienter bei...
\end{itemize}

Ein Vergleich der Pfadlängen und Berechnungszeiten (mit und ohne Eigenkollisionsprüfung) ist in Abbildung \ref{fig:benchmark} dargestellt.

% Beispiel für Bild-Einbindung im HKA Template Stil
\begin{figure}[H]
    \centering
    % Ersetze 'beispiel_plot.png' durch deinen Dateinamen in Figures/
    % \includegraphics[width=0.8\textwidth]{Figures/png/beispiel_plot.png}
    \caption{Vergleich der geplanten Pfade in einer komplexen Umgebung.}
    \label{fig:benchmark}
\end{figure}
	\chapter{Pick-And-Place Szenario}
\label{ch:pickplace}

In diesem Szenario soll der mobile Manipulator einen Block greifen und an einer anderen Position ablegen. Da keine inverse Kinematik verwendet wird, werden die Zielpositionen explizit im Konfigurationsraum vorgegeben.

\section{Erweiterung der Kollisionsprüfung}
Um das "Greifen" zu simulieren, wurde der \textit{CollisionChecker} erweitert. Sobald der Endeffektor das Objekt erreicht, wird das Hindernis (der Block) logisch an das letzte Segment des Roboterarms angehängt.

\section{Ablauf der Simulation}
Der Ablauf gliedert sich in drei Phasen:
\begin{enumerate}
    \item \textbf{Anfahrt:} Der Roboter plant einen Pfad von der Startposition zur Pose des Blocks.
    \item \textbf{Greifen:} Der Block wird Teil der Roboter-Kinematik.
    \item \textbf{Transport:} Der Roboter (nun mit angehängtem Block) plant einen kollisionsfreien Pfad zur Zielablage.
\end{enumerate}

% Hier ggf. eine Bilderserie einfügen (Start -> Griff -> Ende)
	\chapter{Theoretische Konzepte und Erweiterungen}
\label{ch:theorie}

Dieser Abschnitt behandelt weiterführende Fragestellungen der Roboterbahnplanung, die über die Implementierung des planaren Manipulators hinausgehen. Gemäß der Aufgabenstellung werden im Folgenden die notwendigen Systemerweiterungen für translatorische Gelenke sowie Methoden zur nachträglichen Bahnoptimierung diskutiert.





\section{Erweiterung um translatorische Gelenke}
Das in dieser Arbeit betrachtete System besteht aus einer mobilen Basis (3 Freiheitsgrade) und einem Arm mit rotatorischen Gelenken (Revolute Joints). Ein translatorisches Gelenk (Prisma-Gelenk) ermöglicht hingegen eine lineare Relativbewegung zwischen zwei Segmenten, was sowohl den Arbeitsraum als auch die kinematische Modellierung beeinflusst \cite{siciliano2016}.




\subsection{Auswirkung auf die Kinematik}
In der Vorwärtskinematik wird die Lage eines Segments relativ zum vorherigen Segment üblicherweise durch homogene Transformationsmatrizen $T$ beschrieben. 

Für ein \textbf{rotatorisches Gelenk} rotiert das lokale Koordinatensystem um die Gelenkachse (z.\,B. z-Achse) um den variablen Winkel $\theta$. Die Matrix hat die Form:
\begin{equation}
    T_{rot}(\theta) = \begin{bmatrix}
    \cos\theta & -\sin\theta & 0 & 0 \\
    \sin\theta & \cos\theta & 0 & 0 \\
    0 & 0 & 1 & 0 \\
    0 & 0 & 0 & 1
    \end{bmatrix}
\end{equation}

Für ein \textbf{translatorisches Gelenk} ist die Rotation statisch, während die Verschiebung $d$ entlang der Gelenkachse variabel ist. Die Transformationsmatrix ändert sich folglich zu:
\begin{equation}
    T_{trans}(d) = \begin{bmatrix}
    1 & 0 & 0 & 0 \\
    0 & 1 & 0 & 0 \\
    0 & 0 & 1 & d \\
    0 & 0 & 0 & 1
    \end{bmatrix}
\end{equation}

\newpage

\textbf{Implementierungstechnische Konsequenzen:}
Für die Integration in den bestehenden \texttt{CollisionChecker} müsste die Methode zur Berechnung der Robotergeometrie (\texttt{get\_robot\_geometry}) angepasst werden. Anstatt für jedes Gelenk den Winkel zu akkumulieren, müsste für Prisma-Gelenke die effektive Länge des Segments dynamisch berechnet werden. Das Gelenk $q_i$ würde direkt als Längenänderung in die Berechnung der Endpunktkoordinaten eingehen.




\subsection{Anpassung des Konfigurationsraums (C-Space)}
Die Einführung translatorischer Freiheitsgrade verändert die Topologie und die Metrik des Konfigurationsraums \cite{lavalle2006}:
\begin{itemize}
    \item \textbf{Wertebereich und Topologie:} Während rotatorische Gelenke meist periodisch definiert sind ($-\pi$ bis $\pi$), sind translatorische Gelenke durch feste mechanische Anschläge begrenzt ($d_{min}$ bis $d_{max}$). Mathematisch betrachtet ändert sich die Topologie des Gelenks von einem Kreis ($S^1$)\footnote{Die Topologie $S^1$ (1-Sphäre) beschreibt einen kreisförmigen Raum, bei dem die Werte periodisch umlaufen, d.\,h. $-\pi$ und $\pi$ sind identisch.} zu einer Linie ($\mathbb{R}$).
    \item \textbf{Sampling und Metrik:} Wie bereits bei der mobilen Basis ($x, y$ in \ac{LE}, $\theta$ in Radiant), liegen auch bei einem Arm mit translatorischen Gelenken gemischte Einheiten vor. Dies stellt besondere Anforderungen an die Distanzfunktion (Metrik) für die \textit{Nearest-Neighbor}-Suche. Da eine Differenz von $1\,\text{\ac{LE}}$ nicht äquivalent zu $1\,\text{rad}$ ist, muss zwingend eine \textbf{gewichtete Euklidische Metrik} verwendet werden. Hierbei werden die translatorischen und rotatorischen Dimensionen mit Gewichtungsfaktoren $w_i$ skaliert, um einen homogenen Abstandsraum zu schaffen.
\end{itemize}

% \newpage





\section{Optimierung und Glättung von Bewegungsbahnen}
Die von Sampling-basierten Algorithmen erzeugten Pfade bestehen naturgemäß aus linearen Segmenten zwischen zufällig platzierten Wegpunkten. Dies resultiert oft in \enquote{zackigen}, suboptimalen Bahnen mit abrupten Richtungsänderungen. Eine Nachbearbeitung (Post-Processing) ist daher essenziell \cite{choset2005}.




\subsection{Shortcut-Methode (Pruning)}
Ein effizienter Ansatz zur Reduktion der Pfadlänge ist die heuristische Shortcut-Methode. Dieser Algorithmus arbeitet iterativ auf dem initialen diskreten Pfad:

\begin{enumerate}
    \item Es werden zwei zufällige Konfigurationen $q_a$ und $q_b$ auf dem Pfad ausgewählt, die nicht direkt benachbart sind.
    \item Es wird geprüft, ob die direkte Verbindungslinie (Line-of-Sight) im Konfigurationsraum zwischen $q_a$ und $q_b$ kollisionsfrei ist.
    \item Ist die Verbindung valide, werden alle dazwischenliegenden Wegpunkte entfernt und durch die direkte Kante ersetzt.
\end{enumerate}
Das Ergebnis ist ein stückweise linearer Pfad. Dieser ist zwar kürzer, bleibt jedoch lediglich $C^0$-stetig\footnote{$C^0$-Stetigkeit bedeutet, dass der Pfad zusammenhängend ist (keine Sprünge in der Position), aber Knicke aufweist. An diesen Knicken ist die Geschwindigkeit unstetig (der Roboter müsste theoretisch anhalten, um die Richtung zu ändern).}, was zu mechanisch ungünstigen Stopp-and-Go-Bewegungen an den Wegpunkten führt.




\subsection{Glättung mittels Splines}
Um eine kinematisch hochwertige Bahn zu generieren, können die Wegpunkte des optimierten Pfades durch Spline-Kurven approximiert werden. Ziel ist hierbei oft $C^2$-Stetigkeit\footnote{$C^2$-Stetigkeit bedeutet, dass nicht nur die Position und die Geschwindigkeit ($C^1$), sondern auch die Beschleunigung stetig verläuft. Dies ermöglicht ruckfreie Bewegungen.}.

\begin{itemize}
    \item \textbf{B-Splines:} Die Wegpunkte des \ac{PRM}-Pfades dienen hierbei als Kontrollpunkte. Der resultierende Spline verläuft nicht zwingend durch die Punkte, sondern wird von ihnen approximiert. Dies garantiert einen glatten Verlauf, birgt jedoch das Risiko, dass die Kurve Hindernisse schneidet (\enquote{Corner Cutting}).
    \item \textbf{Validierung:} Nach der Spline-Generierung muss die Kurve diskretisiert und erneut auf Kollisionsfreiheit geprüft werden. Sollten Kollisionen auftreten, können iterative Verfahren (z.\,B. Einfügen zusätzlicher Stützpunkte) angewandt werden, um den Pfad lokal zu deformieren und in den freien Raum zurückzuführen.
\end{itemize}

Eine Kombination aus der Shortcut-Methode für die globale Optimierung und anschließender Spline-Interpolation für die lokale Glättung stellt den Stand der Technik für mobile Manipulatoren dar.





\section{Weitere Erweiterungsmöglichkeiten}
Neben den theoretischen Konzepten zur Kinematik und Bahnoptimierung ergeben sich aus den Evaluationen (Kapitel~\ref{ch:evaluation} und~\ref{ch:pickplace}) konkrete Ansätze zur Verbesserung der implementierten Softwarearchitektur und der Planungsstrategien.

\subsection{Optimierung der Kollisionsprüfung}
Wie in Abschnitt~\ref{sec:quantitative_results} (\nameref{sec:quantitative_results}) beschrieben, kann die Reihenfolge der Prüfungen im \textit{CollisionChecker} optimiert werden. Da die Eigenkollisionsprüfung (\textit{Self-Collision}) bedeutend weniger Rechenzeit benötigt als die Sichtbarkeitsprüfung gegen externe Hindernisse, sollte diese vorgezogen werden. Es ist ineffizient, aufwendige Raycasts durchzuführen, wenn die Konfiguration bereits aufgrund einer internen Kollision ungültig ist.

\subsection{Zeit-normalisiertes Benchmarking}
Um die Leistungsfähigkeit der Planer fairer zu vergleichen, könnte ein \textit{Time-Budget}-Ansatz gewählt werden. Dabei erhalten beide Planer (\ac{LazyPRM} und \ac{VisibilityPRM}) exakt die gleiche Rechenzeit. Die Parameter der Planer werden so angepasst, dass sie dieses Zeitfenster optimal nutzen. Anschließend wird die Erfolgsrate (\textit{Success Rate}) als primäres Vergleichskriterium ausgewertet.

\subsection{Entkoppelte Planung}
Zur Steigerung der Geschwindigkeit und Qualität kann die Planung hierarchisch entkoppelt werden. Im ersten Schritt wird lediglich ein Pfad für die Basis ($x, y, \theta_\text{base}$) geplant, während die Armgelenke ($\theta_\text{1}, \theta_\text{2}$) in einer starren Transportkonfiguration verbleiben. Erst in der unmittelbaren Umgebung einer Pick- oder Place-Aktion wird der volle 5-dimensionale Konfigurationsraum betrachtet, um die notwendigen Armbewegungen zu planen.

\subsection{Heuristisches Sampling für LazyPRM}
Der aktuelle \ac{LazyPRM} verteilt neue Punkte in nachfolgenden Iterationen gleichverteilt über den gesamten Raum. Eine effizientere Strategie wäre es, diese Punkte gezielt an Engstellen zu platzieren. Dies kann beispielsweise dadurch erreicht werden, dass bevorzugt in der Nähe von Konfigurationen gesampelt wird, an denen zuvor unzulässige Kanten oder Knoten detektiert wurden, um so kritische Bereiche im Graphen gezielt aufzulösen.
	% \input{TeXFiles/100_Zusammenfassung}

	% \begin{appendix}
	% 	\chapter{Anhang}
\label{ch:Anhang}

\section{Planungsumgebungen}
\label{sec:planungsumgebungen}
% --- TEIL 1 ---
\begin{figure}[H]
    \centering
    \begin{subfigure}[b]{0.48\textwidth}
        \centering
        \includegraphics[width=\textwidth]{Figures/png/Empty_World.png}
        \caption{Empty World}
        \label{fig:empty_world}
    \end{subfigure}
    \hfill
    \begin{subfigure}[b]{0.48\textwidth}
        \centering
        \includegraphics[width=\textwidth]{Figures/png/The_Wall.png}
        \caption{The Wall}
        \label{fig:the_wall}
    \end{subfigure}
    
    \vspace{1em}
    
    \begin{subfigure}[b]{0.48\textwidth}
        \centering
        \includegraphics[width=\textwidth]{Figures/png/Narrow_Passage.png}
        \caption{Narrow Passage}
        \label{fig:narrow_passage}
    \end{subfigure}
    \hfill
    \begin{subfigure}[b]{0.48\textwidth}
        \centering
        \includegraphics[width=\textwidth]{Figures/png/Forest.png}
        \caption{Forest}
        \label{fig:forest}
    \end{subfigure}
    \caption{Benchmarks / Planungsumgebungen}
    \label{fig:planungsumgebungen}
\end{figure}

% --- TEIL 2 ---
\begin{figure}[H]
    \ContinuedFloat
    \centering
    \begin{subfigure}[b]{0.48\textwidth}
        \centering
        \includegraphics[width=\textwidth]{Figures/png/Shelf_Reach.png}
        \caption{Shelf Reach}
        \label{fig:shelf_reach}
    \end{subfigure}
    \caption[]{Benchmarks / Planungsumgebungen (Fortsetzung)}
\end{figure}

\clearpage

\section{Ergebnisse Multi-Run MIT Überprüfung der Eigenkollision}
\label{sec:erg_multiRun_SelfCheck}
% --- TEIL 1 ---
\begin{figure}[H]
    \centering
    \begin{subfigure}[b]{0.48\textwidth}
        \centering
        \includegraphics[width=\textwidth]{Figures/png/Empty_World_benchmark_multiRun_SelfCheck.png}
        \caption{Empty World}
        \label{fig:empty_world_bench_multi}
    \end{subfigure}
    \hfill
    \begin{subfigure}[b]{0.48\textwidth}
        \centering
        \includegraphics[width=\textwidth]{Figures/png/The_Wall_benchmark_multiRun_SelfCheck.png}
        \caption{The Wall}
        \label{fig:the_wall_bench_multi}
    \end{subfigure}
    
    \vspace{1em}
    
    \begin{subfigure}[b]{0.48\textwidth}
        \centering
        \includegraphics[width=\textwidth]{Figures/png/Narrow_Passage_benchmark_multiRun_SelfCheck.png}
        \caption{Narrow Passage}
        \label{fig:narrow_passage_bench_multi}
    \end{subfigure}
    \hfill
    \begin{subfigure}[b]{0.48\textwidth}
        \centering
        \includegraphics[width=\textwidth]{Figures/png/Forest_benchmark_multiRun_SelfCheck.png}
        \caption{Forest}
        \label{fig:forest_bench_multi}
    \end{subfigure}
    \caption{Ergebnisse Multi-Run mit Überprüfung der Eigenkollision}
\end{figure}

% --- TEIL 2 ---
\begin{figure}[H]
    \ContinuedFloat
    \centering
    \begin{subfigure}[b]{0.48\textwidth}
        \centering
        \includegraphics[width=\textwidth]{Figures/png/Shelf_Reach_benchmark_multiRun_SelfCheck.png}
        \caption{Shelf Reach}
        \label{fig:shelf_reach_bench_multi}
    \end{subfigure}
    \caption[]{Ergebnisse Multi-Run mit Überprüfung der Eigenkollision (Fortsetzung)}
\end{figure}

\clearpage

\section{Ergebnisse Multi-Run OHNE Überprüfung der Eigenkollision}
\label{sec:erg_multiRun_noSelfCheck}
% --- TEIL 1 ---
\begin{figure}[H]
    \centering
    \begin{subfigure}[b]{0.48\textwidth}
        \centering
        \includegraphics[width=\textwidth]{Figures/png/Empty_World_benchmark_multiRun_noSelfCheck.png}
        \caption{Empty World}
        \label{fig:empty_world_bench_multi_noCheck} % Label angepasst!
    \end{subfigure}
    \hfill
    \begin{subfigure}[b]{0.48\textwidth}
        \centering
        \includegraphics[width=\textwidth]{Figures/png/The_Wall_benchmark_multiRun_noSelfCheck.png}
        \caption{The Wall}
        \label{fig:the_wall_bench_multi_noCheck} % Label angepasst!
    \end{subfigure}
    
    \vspace{1em}
    
    \begin{subfigure}[b]{0.48\textwidth}
        \centering
        \includegraphics[width=\textwidth]{Figures/png/Narrow_Passage_benchmark_multiRun_noSelfCheck.png}
        \caption{Narrow Passage}
        \label{fig:narrow_passage_bench_multi_noCheck} % Label angepasst!
    \end{subfigure}
    \hfill
    \begin{subfigure}[b]{0.48\textwidth}
        \centering
        \includegraphics[width=\textwidth]{Figures/png/Forest_benchmark_multiRun_noSelfCheck.png}
        \caption{Forest}
        \label{fig:forest_bench_multi_noCheck} % Label angepasst!
    \end{subfigure}
    \caption{Ergebnisse Multi-Run ohne Überprüfung der Eigenkollision}
\end{figure}

% --- TEIL 2 ---
\begin{figure}[H]
    \ContinuedFloat
    \centering
    \begin{subfigure}[b]{0.48\textwidth}
        \centering
        \includegraphics[width=\textwidth]{Figures/png/Shelf_Reach_benchmark_multiRun_noSelfCheck.png}
        \caption{Shelf Reach}
        \label{fig:shelf_reach_bench_multi_noCheck} % Label angepasst!
    \end{subfigure}
    \caption[]{Ergebnisse Multi-Run ohne Überprüfung der Eigenkollision (Fortsetzung)}
\end{figure}

\clearpage

\section{Ergebnisse Multi-Run Pick-And-Place MIT Überprüfung der Eigenkollision}
\label{sec:pp_erg_multiRun_SelfCheck}
% --- TEIL 1 ---
\begin{figure}[H]
    \centering
    \begin{subfigure}[b]{0.48\textwidth}
        \centering
        \includegraphics[width=\textwidth]{Figures/png/Empty_World_benchmark_pickPlace_multiRun_SelfChek.png}
        \caption{Empty World}
        \label{fig:pp_empty_world_bench_multi}
    \end{subfigure}
    \hfill
    \begin{subfigure}[b]{0.48\textwidth}
        \centering
        \includegraphics[width=\textwidth]{Figures/png/The_Wall_benchmark_pickPlace_multiRun_SelfCheck.png}
        \caption{The Wall}
        \label{fig:pp_the_wall_bench_multi}
    \end{subfigure}
    
    \vspace{1em}
    
    \begin{subfigure}[b]{0.48\textwidth}
        \centering
        \includegraphics[width=\textwidth]{Figures/png/Narrow_Passage_benchmark_pickPlace_multiRun_SelfCheck.png}
        \caption{Narrow Passage}
        \label{fig:pp_narrow_passage_bench_multi}
    \end{subfigure}
    \hfill
    \begin{subfigure}[b]{0.48\textwidth}
        \centering
        \includegraphics[width=\textwidth]{Figures/png/Forest_benchmark_pickPlace_multiRun_SelfCheck.png}
        \caption{Forest}
        \label{fig:pp_forest_bench_multi}
    \end{subfigure}
    \caption{Ergebnisse Multi-Run Pick-And-Place mit Überprüfung der Eigenkollision}
\end{figure}

% --- TEIL 2 ---
\begin{figure}[H]
    \ContinuedFloat
    \centering
    \begin{subfigure}[b]{0.48\textwidth}
        \centering
        \includegraphics[width=\textwidth]{Figures/png/Shelf_Reach_benchmark_pickPlace_multiRun_SelfCheck.png}
        \caption{Shelf Reach}
        \label{fig:pp_shelf_reach_bench_multi}
    \end{subfigure}
    \caption[]{Ergebnisse Multi-Run Pick-And-Place mit Überprüfung der Eigenkollision (Fortsetzung)}
\end{figure}

\clearpage

\section{Ergebnisse Multi-Run Pick-And-Place OHNE Überprüfung der Eigenkollision}
\label{sec:pp_erg_multiRun_noSelfCheck}
% --- TEIL 1 ---
\begin{figure}[H]
    \centering
    \begin{subfigure}[b]{0.48\textwidth}
        \centering
        \includegraphics[width=\textwidth]{Figures/png/Empty_World_benchmark_pickPlace_multiRun_noSelfCheck.png}
        \caption{Empty World}
        \label{fig:pp_empty_world_bench_multi_noSelfCheck}
    \end{subfigure}
    \hfill
    \begin{subfigure}[b]{0.48\textwidth}
        \centering
        \includegraphics[width=\textwidth]{Figures/png/The_Wall_benchmark_pickPlace_multiRun_noSelfCheck.png}
        \caption{The Wall}
        \label{fig:pp_the_wall_bench_multi_noSelfCheck}
    \end{subfigure}
    
    \vspace{1em}
    
    \begin{subfigure}[b]{0.48\textwidth}
        \centering
        \includegraphics[width=\textwidth]{Figures/png/Narrow_Passage_benchmark_pickPlace_multiRun_noSelfCheck.png}
        \caption{Narrow Passage}
        \label{fig:pp_narrow_passage_bench_multi_noSelfCheck}
    \end{subfigure}
    \hfill
    \begin{subfigure}[b]{0.48\textwidth}
        \centering
        \includegraphics[width=\textwidth]{Figures/png/Forest_benchmark_pickPlace_multiRun_noSelfCheck.png}
        \caption{Forest}
        \label{fig:pp_forest_bench_multi_noSelfCheck}
    \end{subfigure}
    \caption{Ergebnisse Multi-Run Pick-And-Place ohne Überprüfung der Eigenkollision}
\end{figure}

% --- TEIL 2 ---
\begin{figure}[H]
    \ContinuedFloat
    \centering
    \begin{subfigure}[b]{0.48\textwidth}
        \centering
        \includegraphics[width=\textwidth]{Figures/png/Shelf_Reach_benchmark_pickPlace_multiRun_noSelfCheck.png}
        \caption{Shelf Reach}
        \label{fig:pp_shelf_reach_bench_multi_noSelfCheck}
    \end{subfigure}
    \caption[]{Ergebnisse Multi-Run Pick-And-Place ohne Überprüfung der Eigenkollision (Fortsetzung)}
\end{figure}
	% \end{appendix}
	
	\clearpage

	%% If you want to change the bibliography heading, you can do so by using the following line
	\renewcommand*{\refname}{Literaturverzeichnis}

	%\bibliographystyle{plain} % Numbers in square brackets
	%%\bibliographystyle{alpha} % Initials of first author and year

	% If you want to use the following style, you must activate
	% \usepackage{harvard} % before \begin{document}
	%\bibliographystyle{agsm} % Authors in round brackets
	%\newpage
	\clearpage
	\phantomsection % Otherwise incorrect linking in the table of contents
	\printbibliography[heading=bibintoc]    % Without chapter number/letter
	
	%\printbibliography[heading=bibnumbered]    % With chapter number/letter
	
	\input{TeXFiles/120_Declaration}
\end{document}

%% End of File