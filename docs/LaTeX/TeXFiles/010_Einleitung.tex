\chapter{Einleitung}
\label{ch:einleitung}
In diesem Projekt wird ein mobiler Manipulator simuliert, der aus einer beweglichen Basis (3 Freiheitsgrade) und einem Arm mit zwei rotatorischen Gelenken besteht (insgesamt 5 Freiheitsgrade). Ziel der Arbeit ist die Implementierung eines robusten Kollisionsprüfers (Collision Checker) sowie die Planung und Evaluierung von kollisionsfreien Bewegungsbahnen unter Verwendung von Sampling-basierten Algorithmen. Im Gegensatz zu exakten Verfahren, die den gesamten Konfigurationsraum explizit berechnen müssten, generieren diese Methoden zufällige Stichproben (Samples) der Roboterpositionen, um probabilistisch eine kollisionsfreie Roadmap (einen Graphen) zu erstellen. In dieser Arbeit werden speziell die Varianten 
\begin{listNorm}
    \item \textbf{\ac{LazyPRM}} und
    \item \textbf{\ac{VisibilityPRM}}
\end{listNorm}
betrachtet.
Der erste Teil dieses Berichts, \textbf{Kapitel~\ref{ch:evaluation},~\nameref{ch:evaluation}}, beschreibt die Implementierung des Kollisionsprüfers und die Durchführung von Benchmark-Tests zur Evaluierung der Planungsverfahren in verschiedenen Szenarien.
Zusätzlich wird im zweiten Teil, \textbf{Kapitel~\ref{ch:pickplace},~\nameref{ch:pickplace}}, die Erweiterung zu einer Pick-And-Place Anwendung vorgestellt sowie die Ergebnisse diskutiert. 
Im dritten Teil, \textbf{Kapitel~\ref{ch:theorie},~\nameref{ch:theorie}}, folgen theoretische Konzepte zur Erweiterung des Systems um translatorische Gelenke sowie Möglichkeiten zur Bahnoptimierung.

Die Aufgabe gliedert sich in folgende Teilbereiche:
\begin{listArab}
    \item Implementierung eines \textit{CollisionCheckers} für einen planaren Roboter (Basis + 2 Links).
    \item Benchmarking verschiedener Planungsverfahren (\ac{LazyPRM} vs. \ac{VisibilityPRM}) in unterschiedlichen Szenarien.
    \item Durchführung einer Pick-and-Place Aufgabe, bei der ein Hindernis gegriffen und transportiert wird.
    \item Theoretische Betrachtung von Systemerweiterungen (translatorische Gelenke und Pfadglättung).
\end{listArab}

Der vollständige Quellcode dieses Projekts ist im entsprechenden Git-Repository hinterlegt: \url{https://github.com/Paulchenn/Mobiler_Manipulator}. Für detaillierte Anweisungen zur Installation und Verwendung wird auf die dort befindliche \texttt{README}-Datei verwiesen.
