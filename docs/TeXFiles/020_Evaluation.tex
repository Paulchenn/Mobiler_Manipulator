\chapter{Implementierung und Evaluation}
\label{ch:evaluation}

\section{Der CollisionChecker}
\label{sec:collision}

Die Basis für alle Planungsverfahren ist eine zuverlässige Kollisionserkennung. Der Roboter wird hierbei modelliert durch eine rechteckige Basis und einen Arm mit zwei Segmenten.

\subsection{Modellierung}
Die Geometrie des Roboters wird über eine Liste definiert, welche die Längen und Dicken der Segmente sowie die Gelenkwinkelgrenzen enthält. Ein typisches Segment ist beispielsweise definiert als:
\begin{lstlisting}
# Beispiel eines Listenelements: [Laenge, Dicke, [MinWinkel, MaxWinkel]]
segment = [5.1, 1, [-3.14, 3.14]] 
\end{lstlisting}

\subsection{Umgang mit Eigenkollisionen}
Eine besondere Herausforderung stellt die Eigenkollision (Self-Collision) dar. Hierbei muss verhindert werden, dass der Arm des Roboters die eigene Basis oder andere Armsegmente durchdringt. Dies wurde implementiert, indem...
% Hier kurz erklären, wie du das gelöst hast (z.B. Prüfung aller Segment-Paare).

\section{Benchmarking der Planer}
Zur Bewertung der Algorithmen \textit{LazyPRM} und \textit{VisibilityPRM} wurden 5 Benchmark-Umgebungen mit steigendem Schwierigkeitsgrad erstellt.

\subsection{Ergebnisse}
Wie im Notebook \texttt{IP-X-1-Automated\_PlanerTest.ipynb} analysiert, zeigen sich folgende Unterschiede:
\begin{itemize}
    \item \textbf{LazyPRM}: Zeigt Stärken in...
    \item \textbf{VisibilityPRM}: Ist effizienter bei...
\end{itemize}

Ein Vergleich der Pfadlängen und Berechnungszeiten (mit und ohne Eigenkollisionsprüfung) ist in Abbildung \ref{fig:benchmark} dargestellt.

% Beispiel für Bild-Einbindung im HKA Template Stil
\begin{figure}[H]
    \centering
    % Ersetze 'beispiel_plot.png' durch deinen Dateinamen in Figures/
    % \includegraphics[width=0.8\textwidth]{Figures/png/beispiel_plot.png}
    \caption{Vergleich der geplanten Pfade in einer komplexen Umgebung.}
    \label{fig:benchmark}
\end{figure}